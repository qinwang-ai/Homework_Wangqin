\documentclass[a4paper]{article}
\usepackage{listings}
\usepackage{xcolor}
\usepackage {fontspec}
\setromanfont{Lantinghei SC Extralight}
\setmonofont{Courier New}
\XeTeXlinebreaklocale ``zh''
\XeTeXlinebreakskip = 0pt plus 1pt
\textheight = 650pt
\lstset{
	%行号
   numbers=left,
  %背景框
   framexleftmargin=10mm,
   frame=none,
   %背景色
   %backgroundcolor=\color[rgb]{1,1,0.76},
   backgroundcolor=\color[RGB]{245,245,244},
   %样式
   keywordstyle=\bf\color{blue},
   identifierstyle=\bf,
   numberstyle=\tiny,
   numberstyle=\color[RGB]{0,192,192},
   commentstyle=\it\color[RGB]{0,96,96},
   stringstyle=\rmfamily\slshape\color[RGB]{128,0,0},
   %显示空格
   showstringspaces=false
 }

\begin{document}
\title{实验报告 实验五}
\author{姓名:王钦\quad 学号:13349112\quad 班级:计科二班}
\date{}

\maketitle
\section*{ 实验目的}
\hangindent=4em \hangafter=-10{
  1. 理解系统调用的实现方法。\\
  2. 实现原型操作系统中一些基本的系统调用。\\
  3. 设计并实现一测试系统调用的用户程序,利用系统调用实现用户界面和内部功能。\\
  4.在原型操作系统上建立一个初步C语言开发环境,理解操作系统与高级语言之间的关系。\\
}
\section*{ 实验内容}
\hangindent=4em \hangafter=-10{
在实验四的基础上,进化你的原型操作系统,增加下列操作系统功能:\\
(1)参考下面的系统调用功能表,增加一些其他功能\\
系统调用表(局部设计)\\
(2)扩展MYOS内核,实现上表中的所有(包括你增加的)系统调用,并开发一个用户程序,展示这些系统调用的使用效果。\\
(3)设计一个C程序库,封闭getch(),gets(),putch(),puts(),scanf()和printf()等利用系统调用实现的细节,并参考下面程序,开发一个用户程序,测试这些函数功能。\\

{\scriptsize
\begin{lstlisting}[language={C}]
include stdio.h
main(){
   	char ch,str[80];
   	int a;
   	getch(&ch);
   	gets(str);
   	scanf(“a=%d”,&a);
  	putch(ch);
  	puts(str);
   	printint(“ch=%c, a=%d, str=%s”, ch, a, str);
}
\end{lstlisting}
}

\section*{ 实验平台}
\hangindent=4em \hangafter=-10{
  gcc+ld+nasm+Linux+vim\\
}

\section*{ 算法流程图}
\hangindent=4em \hangafter=-10{
  \includegraphics[scale=0.5]{Illustrations/flow.png}
}

\section*{ 实验步骤及效果}
\hangindent=4em \hangafter=-50{
1. 编辑修改ASM 文件,和C文件	\\\\
2. 使用make命令配合\verb|makefile|文件进行编译	\\\\
3. 运行\verb|bochs or vmware|虚拟机进行测试,进入后所看到的欢迎界面
{\center\includegraphics[scale=0.5]{Illustrations/start.png}}\\\\
4. 这次比实验四多了一个系统内置程序\verb|python|,我们可以执行\verb|man python |\\查看他的作用,主要功能是输入一个数学表达式然后返回表达式的结果,类似\verb|python|命令行的作用。
但目前这个工具只支持加法和减法且只能有两个操作数(其实主要为了展示\verb|ah=3,4|将字符串转为数值和将数值转为字符串的系统调用效果),我这里仿照\verb|linux|系统的系统调用,设置
\verb|int 80h |为所有系统调用的入口。可以看到图中分别输入加法减法,返回计算结果。如果输入的不符合格式就会返回错误提示。最后输入\verb|exit|退出\verb|python|命令行工具。
{\center\includegraphics[scale=0.5]{Illustrations/python.png}}\\\\
5. 接下来测试ah等于\verb|0,1,2,5|的系统调用(显示\verb|OUCH|,字母大小写变化等),输入\verb| run 2 |,运行第二个用户程序,
{\center\includegraphics[scale=0.5]{Illustrations/syscall0125.png}}\\\\

下面是第二个用户程序的代码:
{\scriptsize
	\begin{lstlisting}[language={C}]
org 0x1000
;org 0x100

;#0
mov ah,0
int 80h

;LISTEN_EXIT----
listen0:
	mov ah,0
	int 16h


;#2
mov ax,0xb800
mov es,ax
mov dx,1994D
mov ah,2
int 80h

;LISTEN_EXIT----
	mov ah,0
	int 16h

;#1
mov ax,0xb800
mov es,ax
mov dx,1994D
mov ah,1
int 80h

;LISTEN_EXIT----
	mov ah,0
	int 16h



;#5
mov ax,cs
mov ds,ax
mov es,ax

mov cx,0317h  ;position
mov ah,5
mov dx,msg
int 80h


;LISTEN_EXIT----
listen:
	mov ah,0
	int 16h

ret



msg:
	db "hello world!"

times 512-($-$$) db 0	;填充剩余扇区0



	\end{lstlisting}}
下面我们将上面代码分解一下,详细介绍。\\\\
5.1 首先执行执行\verb|ah=0|的系统调用将\verb| OUCH |  打印在屏幕中间
{\scriptsize
\begin{lstlisting}[language={C}]
org 0x1000
;org 0x100

;#0
mov ah,0
int 80h
;LISTEN_EXIT----
listen0:
	mov ah,0
	int 16h

	\end{lstlisting}}
{\center\includegraphics[scale=0.4]{Illustrations/syscall0.png}}\\\\
5.2 现在按下任意键,将执行\verb|ah=2|的系统调用把\verb| OUCH|的第一个字母大写O变成小写o\\\\
{\scriptsize
\begin{lstlisting}[language={C}]
;#2
mov ax,0xb800
mov es,ax
mov dx,1994D
mov ah,2
int 80h

;LISTEN_EXIT----
	mov ah,0
	int 16h

	\end{lstlisting}}
{\center\includegraphics[scale=0.4]{Illustrations/syscall2.png}}\\\\
5.3 按下任意键,执行\verb|ah=1|的系统调用把\verb| oUCH|的第一个字母小写o变回大写O\\\\
{\scriptsize
\begin{lstlisting}[language={C}]
;#1
mov ax,0xb800
mov es,ax
mov dx,1994D
mov ah,1
int 80h

;LISTEN_EXIT----
	mov ah,0
	int 16h


	\end{lstlisting}}
{\center\includegraphics[scale=0.4]{Illustrations/syscall1.png}}\\\\
5.4 按下任意键,执行\verb|ah=5|的系统调用在屏幕3行17列的位置打印一个\verb|helloworld|
{\scriptsize
\begin{lstlisting}[language={C}]
;#5
mov ax,cs
mov ds,ax
mov es,ax

mov cx,0317h  ;position
mov ah,5
mov dx,msg
int 80h


;LISTEN_EXIT----
listen:
	mov ah,0
	int 16h

ret

msg:
	db "hello world!"
	\end{lstlisting}}

{\center\includegraphics[scale=0.4]{Illustrations/syscall5.png}}\\\\
5.5 返回操作系统\\\\
6. 接下来我们测试封装的osclib\_share.c库里面实验所要求的\\
\verb|getch,gets,scanf,putch,putch,printint| ,输入\verb|run 1|执行第一个用户程序\\
代码如下:\\
{\scriptsize\begin{lstlisting}[language={C}]
void main(){
	screen_init();
	getch( &ch);
	gets( key);
	putch('\r');
	putch('\n');
	scanf("a=%d",&a);

	putch( ch);
	puts( key);
	printint( "ch=%c,a=%d,str=%s",ch,a,key);

	wait_key();
}

\end{lstlisting}}
6.1 首先程序等待输入一个字符
{\center\includegraphics[scale=0.4]{Illustrations/usr1_wait.png}}\\\\
6.2 输入字符 C,回车再输入一个字符串\verb|I am a string|回车
{\center\includegraphics[scale=0.4]{Illustrations/usr1_inputstr.png}}\\\\
6.3 再输入一个int类型的数字16,回车后执行下面代码\\
{\scriptsize\begin{lstlisting}[language={C}]
	putch( ch);
	puts( key);
	printint( "ch=%c,a=%d,str=%s",ch,a,key);
\end{lstlisting}}
可以看到相继打印出了之前输入的内容
{\center\includegraphics[scale=0.4]{Illustrations/usr1_over.png}}\\\\

7. 和之前的实验一样,内核也有设置中断,系统内置程序和重复运行多个用户程序,这里就不再赘述了。
{\center\includegraphics[scale=0.4]{Illustrations/runseq.png}}\\\\
{\center\includegraphics[scale=0.4]{Illustrations/allcustomint.png}}\\\\
{\center\includegraphics[scale=0.4]{Illustrations/usr3.png}}\\\\

\section*{ 内存和软盘存储管理}
\hangindent=4em \hangafter=-50{
1. 引导程序加载到内存0x7c00处运行\\
2. 引导程序将操作系统加载到0x7e00处运行\\
3. 操作系统讲用户程序加载到0x1000处运行\\
4. 软盘第1个柱面的第一个扇区存储操作系统引导程序\\
5. 软盘第1个柱面剩下所有扇区2~36扇区存储操作系统内核\\
6. 软盘第\verb|2,3,4|柱面分别存储三个用户程序\\
更多细节信息请阅读我的Makefile文件
}
\section*{ 主要函数模块解释}
\hangindent=4em \hangafter=-50{
	内核架构解释:\\
	\verb|os.c|为内核主要控制模块\\
	\verb|osclib.c os.asm|主要为\verb|os.c|提供函数实现.\\
	\verb|oslib.asm| 为\verb|osclib.c|提供更底层的函数封装\\
	\verb|osclib_share.c,oslib_share.asm| 为用户程序中所需要用到的函数,从\verb|osclib.c oslib.asm|中取出一部分作为内核和用户的共享库(使用户程序体积减少)。\\
	\verb|os_syscall.asm| 初始化系统调用和设置系统调用相关模块\\
	更多细节信息请阅读我的Makefile文件\\\\

	1. \verb|os.c: main| 函数模块,这个在之前报告中已经解释这里就不在赘述
{\scriptsize\begin{lstlisting}[language={C}]
void main(){
	//--------------------init
	init_ss();
	screen_init();
	interrupt_init();
	syscall_init();
	print_welcome_msg();
	print_message();
	print_flag(); //root@wangqin4377@:   position
	//--------------------init_end

	while(1){
		char length = listen_key();
		if_screen_scroll(); //bottom of screen
		flag_scroll();//move flag to next line
		print_flag();
	}
}
 \end{lstlisting}}
	2. 本次试验主要为了实现系统调用的工作,故在 os.asm中实现了下列函数供设置系统调用使用
{\scriptsize
	\begin{lstlisting}[language={C}]
	;---PARAM: ah is syscall num   ebx is address of syscall  bx:temp cx:function ax:sysnum
	setting_up_syscall:
	mov bx,0
	mov es,bx
	mov al,ah
	mov ah,0
	shl al,2
	mov bx,0xfe00
	add bx,ax
	mov [es:bx],ecx
	ret
	\end{lstlisting}}
	3. 初始化设置系统调用,设置全部6个系统调用功能,调用上面的函数实现

{\scriptsize
	\begin{lstlisting}[language={C}]
syscall_init:

;----#0 syscall
mov ah,0
mov ecx,0
mov cx,display_center_ouch
call setting_up_syscall

;----#1 syscall
mov ah,1
mov ecx,0
mov cx,letter_upper
call setting_up_syscall

;----#2 syscall
mov ah,2
mov ecx,0
mov cx,letter_lower
call setting_up_syscall

;----#3 syscall
mov ah,3
mov ecx,0
mov cx,atoi_syscall
call setting_up_syscall

;----#4 syscall
mov ah,4
mov ecx,0
mov cx,itoa_syscall
call setting_up_syscall

;----#5 syscall
mov ah,5
mov ecx,0
mov cx,display_str
call setting_up_syscall
ret
	\end{lstlisting}}
	4. 对\verb|scanf,printint,gets..|等函数的封装:
	其实这些函数在之前的实验中已经实现,只是名字不一样罢了。本次实验单独抽出来相关代码放在osclib\_share.c中。
	{\scriptsize
	\begin{lstlisting}[language={C}]
	\end{lstlisting}}
	5. python命令行工具:存放在\verb|python_extension.c|中,主要功能的就是使用系统调用实现字符串和数值之间的转换。
	\begin{lstlisting}[language={C}]
	unsigned short int itoa_temp;
	char * itoa_ans;
	char * itoa( short int x){
		itoa_temp = x;
		__asm__("mov $4,%ah");  // syscall num
		__asm__("push %bp");
		__asm__("int $0x80");	// syscall
		__asm__("pop %bp");
		return itoa_ans;
	}

	char *atoi_temp;
	unsigned short int atoi_ans;
	unsigned short int atoi( char * str){
		atoi_temp = str;
		__asm__("mov $3,%ah");  // syscall num
		__asm__("int $0x80");	//syscall
		return atoi_ans;
	}
	\end{lstlisting}
}

\section*{ 实验心得及仍需改进之处}
\hangindent=4em \hangafter=-50{
    实验心得:\\\\
	\indent	通过本次试验我手动编写仿照linux的系统调用通过设置功能号ah然后使用int 80h中断来调用系统服务,让我了解了用户调用内核的系统服务的机制和原理
	在实验的过程中遇到了很多问题,比如设置好系统服务程序后,使用\verb|int 80h| 调用的时候却无法工作。通过使用bochs来调试解决了。
	\\ \indent 总结一下所有的bug分为两类:一类是因为自己缺乏相关知识或经验错误的使用了一些指令导致的,另一类就是自己粗心大意,在细节上没有处理好,结果调试很久才发现是一个小细节上疏忽了。对我
	而言后一类是经常遇到的,以后的实验一定在编写代码的时候一定要非常谨慎,必须每一步都要考虑操作系统全局的实现,必须要先规划好层次和模块再去实现,不能茫然的
	直接开始写代码。其中如何将实验要求的实现系统服务功能展示出来我就换了很多种方式,最后确定功能好为\verb|3,4| 的在python 命令行工具使用的过程中展示,其他的
	在用户程序2中展示。
	\\ \indent 在做封装c的一些函数的时候,因为之前实验已经做好了那些函数,但是这些函数都和庞大的\verb|osclib.c os.asm oslib.asm|密不可分,我就直接将这三个很大的
	类库与用户1的程序联合编译,结果可想而知导致用户1的程序非常大,只能改变用户程序的存储方式,原来是一个扇区存储一个用户程序,改为一个柱面存储一个用户程序。后来
    我将用户1程序中c函数所用到的一些代码从庞大的三个文件中分离出来,建立一个操作系统和用户的共享库\verb|osclib_share.c oslib_share.asm|。使得用户1的体积减小了五倍,
	只占两个扇区。希望在以后的实验中能实现在内核中直接可以解析执行elf文件,这样用户程序就可以已elf的格式存在,动态链接操作系统的类库,而不是自己也带一份一模一样的。\\\\
	实验仍需改进之处:\\\\
	仍需完善细节,比如说python的命令行工具加入乘法,除法完全是几行代码的问题\\
	可以考虑将用户程序做成elf格式,动态链接系统的代码库。目前的情况是用户程序自己带着一份和操作系统一样的代码库,分别联合编译\\
	考虑精简操作系统代码,减少冗余代码\\
	继续优化调整内核架构和内存磁盘管理\\
}

\end{document}
