\documentclass[a4paper]{article}
\usepackage {changepage}
\usepackage{fancyhdr}
\usepackage {fontspec}
\usepackage {paralist}
\usepackage {multicap}
\pagestyle{fancy}
\setromanfont{Lantinghei SC Extralight}
\setmonofont{Courier New}
\XeTeXlinebreaklocale ``zh''
\XeTeXlinebreakskip = 0pt plus 1pt
\textheight = 650pt
\begin{document}
\title{实验报告 Lab 6}
\author{姓名:王钦\quad 学号:13349112}
\date{}
\maketitle

\section*{ Part I: Understanding minisniff}
\hangindent=4em \hangafter=-200{
	\begin{enumerate}
		\item use pcap libaray.I can find more information on http://www.tcpdump.org/
		\item Advantage
		\begin{itemize}
			\item easy to use
			\item compatible with other programs
			\item compatible with operat system both windows and linux
			\tiem capture all incoming and outgoing packets
		\end{itemize}
		\item Disadvantage
		\begin{itemize}
			\item writen on C language,not have python java etc. other language version.
			\item if dont have computer basic knowledge. it's difficult to learn.
		\end{itemize}
		\item Yes,I have searched on github and find some open source repositories like this one \verb|https://github.com/AnwarMohamed/Packetyzer|
		\item Explain functions(use man command on linux)
		  \begin{itemize}
			  \item  \verb|pcap_lookupdev| - find the default device on which to capture
			  \item \verb|pcap_open_live| - open a device for capturing
			  \item \verb|pcap_lookupnet| - find the IPv4 network number and netmask for a device
			  \item \verb|pcap_compile| - compile a filter expression
			  \item \verb|pcap_setfilter| - set the filter
			  \item \verb|pcap_next| - read the next packet from a \verb|pcap_t|
			  \item \verb|pcap_loop| - process packets from a live capture or savefile
			  \item \verb|pcap_dispatch| - process packets from a live capture or savefile
		  \end{itemize}
		\item on application layer
	\end{enumerate}
}
\section*{ Part II: Extending minisniff}
\hangindent=4em \hangafter=-200{
	\begin{enumerate}
		\item 
	\end{enumerate}
\end{document}




